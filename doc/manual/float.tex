
\chapter{Floating point}
\label{floatingpoint}

Starting with version 5.0 \FORM\ is equiped with arbitrary floating point 
capability. The low level routines are part of the GMP and mpfr libraries 
which should be available on most systems. If not they can be picked up 
easily from the internet. The main commands involving the floating point 
system are
\begin{description}
\item[\#startfloat] This instruction is needed to startup the floating 
point system. Invoking it will allocate a number of arrays. The instruction 
has either one or two arguments:
\begin{verbatim}
    #startfloat <precision> [,MZV=<maximumweight>]
\end{verbatim}
The first argument is mandatory and specifies the desired precision. It must 
be a positive integer followed by either \texttt{b} (for precision in bits) 
or \texttt{d} (for precision in decimal digits).
\FORM{} will round to at least this precision. Because the internal 
routines work with WORDs, the precision (in bits) will internally be rounded up to the nearest 
integer number of WORDs. The second argument is optional for when one wants 
to work with multiple zeta values (MZVs) or Euler sums. It specifies the 
maximum weight that will be used. The evaluation of the sums requires a 
number of auxiliary arrays. The default value is zero. If one would like to 
change the precision during a run, this is possible. The effect would be 
that the existing arrays are released and new arrays will be allocated.
\item[\#endfloat] This instruction releases all arrays allocated for the 
floating point system.
\end{description}
\begin{description}
\item[tofloat] Converts the rational coefficients at the ground level to 
floating point numbers in the precision specified in the \#startfloat 
instruction. From this point on the coefficient at this level will be 
floating point. If one needs to convert numbers inside a function argument 
one should use the argument environment. This can be nested.
\item[torational] Tries to convert the floating point coefficients to 
rational numbers. To this end it uses repeated fractions as in
\begin{eqnarray}
	x & \rightarrow & n_0 + 1/(n_1+1/(n_2+1/(n_3+\cdots))) \nonumber
\end{eqnarray}
with $x$ a floating point number. The algorithm keeps track of the 
remaining precision and if $1/n_i$ is close to this precision it truncates 
the sequence at $n_{i-1}$. After that it works out the fraction. It could 
be that $x$ cannot be expressed as a fraction within the given precision. 
This can usually be seen by that the fractions are `rather wild', or that 
the result changes when the precision is increased. This statement can also 
be abbreviated to `torat'.
\item[evaluate] If this command has no arguments all floating point 
functions that \FORM{} knows about will be evaluated. The currently allowed 
arguments are the functions mzv\_, euler\_, sqrt\_ and mzvhalf\_. If any 
(or more than one) of these are specified only those functions will be 
evaluated.
\item[strictrounding] This statement rounds floating point numbers to a 
given precision. The syntax is
\begin{verbatim}
    strictrounding [precision];
\end{verbatim}
where precision is an optional argument that specifies the rounding 
precision in either digits or bits, using the same syntax as 
\texttt{\#startfloat}. If no argument is given, this statement rounds 
the floating point coefficients to the default precision. Internally, 
the GMP and mpfr libraries may use extra precision beyond that set by 
\texttt{\#startfloat}. As a result, terms may not merge due to this 
extra precision. For example:
\begin{verbatim}
    #startfloat 6d
    CFunction f;
    Local F = f(1.0)+f(1.0000001);
    Print;
    .sort
\end{verbatim}
results in \texttt{F = f(1.0e+00)+f(1.0e+00);}. Although it may appear 
that the terms should merge, the extra precision maintained by GMP 
prevents this, even though it is not displayed at 6 digits of 
precision. Using the strictrounding statement, one can force rounding 
to exactly 6 digits. Indeed:
\begin{verbatim}
    Argument f;
        strictrounding;
    Argument;
    Print;
    .end
\end{verbatim}
results in \texttt{F = 2*f(1.0e+00);}.
Notice that rounding in bits may produce unexpected results when viewed 
in decimal digits. For example, the decimal number 1.1e-4 cannot be 
represented exactly in binary. Its binary representation, up to 20 bits of precision, is 
$1.1100110101011111101*2^{-14}$. When rounded to 5 bits, this becomes 
$1.1101*2^{-14}$, which in decimal digits appears as 
1.10626220703125e-04.
\item[Format floatprecision] This instruction controls how many digits are 
displayed when printing floating point numbers. It only affects output 
formatting and does not influence the internal precision or accuracy of 
computations. It can be used in three ways: in case of 
\begin{verbatim}
    Format floatprecision;
\end{verbatim}
\FORM{} prints floats with the number of digits specified by the current 
\#startfloat instruction. With
\begin{verbatim}
    Format floatprecision <precision>;
\end{verbatim}
\FORM{} prints the number of digits specified by \texttt{<precision>}. 
The syntax is the same as for the precision in \#startfloat: a positive 
integer followed by either \texttt{b} (for bits) or \texttt{d} (for decimal 
digits). If the requested precision exceeds the precision specified by 
\#startfloat, only the available digits are printed. Finally, with 
\begin{verbatim}
    Format floatprecision off;
\end{verbatim}
the floating point numbers are printed in raw internal format. 
\end{description}
In addition to the above commands there are the following functions that 
can be evaluated sqrt\_, ln\_, li2\_, gamma\_, agm\_, sin\_, cos\_, tan\_,
asin\_, acos\_, atan\_, sinh\_, cosh\_, tanh\_, asinh\_, acosh\_, atanh\_.
For the functions atan2\_ and lin\_ there is currently no code.
The agm\_ function is the arithmetic geometric mean of its two input 
values.

In addition to the above functions there are also the constant 
pi\_\index{pi\_}, the basis of the natural logarithm ee\_\index{ee\_} and the 
Euler-Mascheroni constant em\_\index{em\_}. These constants will also be 
expanded with the evaluate command. When given as an argument to evaluate, 
only the specified constants will be evaluated.

The following example shows some work with Multiple Zeta Values (MZV's):
\begin{verbatim}
    #StartFloat 500b, MZV=15
    L F1 =
    	-mzv_(8,1,1,5)
    	+29056868/39414375*mzv_(2)^6*mzv_(3)
    	-47576/40425*mzv_(2)^5*mzv_(5)
    	-163291/18375*mzv_(2)^4*mzv_(7)
    	-4/105*mzv_(2)^3*mzv_(3)^3
    	-450797/11025*mzv_(2)^3*mzv_(9)
    	+7/5*mzv_(2)^2*mzv_(3)^2*mzv_(5)
    	+16/25*mzv_(2)^2*mzv_(3)*mzv_(5,3)
    	+454049/1400*mzv_(2)^2*mzv_(11)
    	-16/25*mzv_(2)^2*mzv_(5,3,3)
    	+3*mzv_(2)*mzv_(3)^2*mzv_(7)
    	+61/14*mzv_(2)*mzv_(3)*mzv_(5)^2
    	+2/7*mzv_(2)*mzv_(3)*mzv_(7,3)
    	+2172853/420*mzv_(2)*mzv_(13)
    	-2/7*mzv_(2)*mzv_(7,3,3)
    	+1/7*mzv_(2)*mzv_(5,5,3)
    	-33/4*mzv_(3)^2*mzv_(9)
    	-133/6*mzv_(3)*mzv_(5)*mzv_(7)
    	-25/9*mzv_(3)*mzv_(9,3)
    	-244/105*mzv_(5)^3
    	-359/105*mzv_(5)*mzv_(7,3)
    	+3/10*mzv_(7)*mzv_(5,3)
    	+89/18*mzv_(9,3,3)
    	+569/105*mzv_(7,3,5);
    L F2 = mzv_(15);
    Evaluate mzv_;
    Print +f;
    .sort

   F1 =
       9.1234206877960755900164875575406726239325002222490534540605137258846994\
       916348297032751308227224952419629422497720599224543719959652966613231560\
       6913926e+03;

   F2 =
       1.0000305882363070204935517285106450625876279487068581775065699328933322\
       671563422795730723343470175484943669684442492832530297757588781904321794\
       40477e+00;

    Skip F1,F2;
    L	X = F1/F2;
    Torational;
    Print +f;
    .end

   X =
      229903169/25200;

  0.08 sec out of 0.09 sec
\end{verbatim}
The \#startfloat initializes the floating point system and allocates arrays 
for 500 bits of precision. If there is a second number it indicates the 
maximum weight for MZVs and Euler sums. The functions are only evaluated 
when the proper command is given. In the second module we divide the 
numbers and convert the result to a rational. It is a good idea to try this 
with various precisions to see whether this is stable. With 60 bits the 
final answer would be
\begin{verbatim}
    145537942402475031/15952490572234;
\end{verbatim}
while at 150 bits we have already the same answer as with 500 bits. The 
fraction that is obtained by this program can be proven to be correct.
\vspace{3mm}

