
\chapter{Output optimization}
\label{optimization}

One of the uses of symbolic programs is to prepare formulas for further 
numerical processing\index{numerical processing}. Technically speaking such 
processing is not part of computer algebra, although some packages may 
provide facilities for this. In \FORM\ such facilities, such as
Monte Carlo integration, do not exist at the moment, but, starting with 
version 4.1, \FORM\ does provide statements to construct outputs in C or 
Fortran that are highly optimized with respect to the number of arithmetic 
operations\index{arithmetic operations} that are needed for their 
evaluation. The algorithms used for this are described in the papers
\begin{itemize}
  \item Code Optimization in FORM - \url{https://arxiv.org/abs/1310.7007}
  \item Improving multivariate Horner schemes with Monte Carlo tree search - \url{https://arxiv.org/abs/1207.7079}
  \item Combining Simulated Annealing and Monte Carlo Tree Search for Expression Simplification - \url{https://arxiv.org/abs/1312.0841}
  \item Why Local Search Excels in Expression Simplification - \url{https://arxiv.org/abs/1409.5223}
\end{itemize}
In short, an optimal Horner scheme is constructed after
which common subexpressions are eliminated. The methods for finding the
optimal scheme can use a simple heuristic, Monte Carlo Tree Search,
or a Stochastic Local Search approach such as Simulated Annealing

In this section the precise 
format of the commands that concern the optimizations will be described.
In optimized output \FORM\ needs temporary variables\index{temporary 
variables}. In order to avoid conflicts with user defined objects \FORM\ 
uses the extra symbols \ref{substaextrasymbols}\index{extra symbols} for 
these variables. This means that the user can control their output 
representation in the standard way. In addition there are preprocessor 
variables that tell how many of these extra symbols were needed:
\begin{description}
\item[optimminvar\_] The number of extra symbols before the optimization 
process started\index{optimminvar\_}.
\item[optimmaxvar\_] The number of extra symbols after the optimization 
process finished\index{optimmaxvar\_}.
\end{description}
Each new optimization will remove the old optimization results and start 
the extra symbols from the number there were before the optimization 
started. Because this may cause interference with the functioning of the 
extrasymbol statement, regular printing with output optimization and the 
extrasymbol statement cannot occur inside the same module. Such occurrence 
would result in an error message.

Because the output optimization is done for expressions that contain only 
symbols\index{symbols}, \FORM\ has to convert all non-symbols and negative 
powers of symbols to extra symbols\index{extra symbols} before it starts 
the optimization. This is another reason why interference between the 
extrasymbol \ref{substaextrasymbols}\index{extra symbols} statement and 
output optimizations is forbidden. When the results are printed, the 
definition of the extra symbols that are introduced this way are printed as 
well.

\FORM\ has two ways to perform optimizations. The first and easiest is in 
the regular output. If one asks for optimization (by specifying the proper 
format for this) and follows this by a print statement, the output printed 
will be in optimized form. This is however just a representation of the 
expression and the next module will obtain the original expression for its 
input.

The more useful way to obtain an optimized output is with the \#optimize 
instruction. To use this instruction properly one should understand what 
\FORM\ does when it optimizes an expression. The whole process of 
optimization takes place inside the memory. Hence, \FORM\ cannot optimize 
expressions that do not fit inside the CPU memory. The notation is however 
fairly compact and \FORM\ needs far less space than for instance the 
compiler (and gives better results). The result of the optimization is 
stored inside a buffer. There is only a single optimization 
buffer\index{optimization buffer} and the preprocessor variables 
optimminvar\_\index{optimminvar\_} and optimmaxvar\_\index{optimmaxvar\_} 
refer to the contents of this buffer. When the \#optimize instruction is 
used it loads this buffer and the contents stay around until either a 
\#clearoptimize instruction is used or a new \#optimize instruction is 
issued.

The \#optimize instruction changes the original expression to its optimized 
shape in which it is usually a very short expression that refers to one or 
more extra symbols. The optimization information is automatically erased, 
and with it the expression that was optimized, when a second \#optimize 
instruction is issued. Clearing the optimization buffer means that the 
information of the first expression is irretrievably lost and the contents 
of the first expression become meaningless, because its extra symbols have 
been erased. Hence if the user still needs this expression it is necessary 
to make a copy of it before optimization.

The optimization buffers, and the optimized expression, can be removed by 
the user with the \#clearoptimize instruction. This is mandatory before the 
use of a ToPolynomial \ref{substatopolynomial}\index{ToPolynomial} 
statement, because that may introduce new extra symbols.

The contents of the optimization buffer\index{optimization buffer} can be 
written with the \%O combination in the format string in the \#write 
instruction. This means that it is easy to write this output to file. 
Consider for instance the following program:
\begin{verbatim}
   CF  f;
   S   a,b,c;
   L   H = f(a)+f(b)+(a+b+c)^2;
   L   G = f(c)+(a+b+c)^3;
   Format O2;
   Print +f;
   .sort
   ExtraSymbols,array,w;
   Format Fortran;
   #optimize G
   #write <outg.f> "      REAL*8 w(`optimmaxvar_')"
   #write <outg.f> "%O"
   #write <outg.f> "      G = %e",G
   #clearoptimize
   .sort
   #optimize H
   #write <outh.f> "      REAL*8 w(`optimmaxvar_')"
   #write <outh.f> "%O"
   #write <outh.f> "      H = %e",H
   .end
\end{verbatim}
This program shows the two different methods and shows what is left of the 
expressions G and H. It also shows that we have to deal with the 
expressions one by one when we use the \#optimize instruction, while in the 
regular printing of the output this is not needed because the expression 
itself remains in its unoptimized version.

\subsection{Optimization options of the Format statement}

The \verb|Format| statement has a number of options to control the
code optimization. The easiest to use are the following:

\begin{description}
\item[O0] Switches off all optimizations and prints the output the
  normal \FORM\ way. This is the default.

\item[O1] Activates the lowest level of optimization. It is very fast,
  i.e., linear in the size of the expression, and gives reasonably
  efficient code.

\item[O2] Activates the medium level of optimization. This is slower
  than the previous setting, but usually gives better results.

\item[O3] Activates the highest level of optimization using MCTS. It can be
  rather slow, but usually gives even better results.

\item[O4] Activates the highest level of optimization using Local Stochastic Search. 
It is usually much faster than MCTS and may give better results.
\end{description}

Below we show how to use O4 and how it compares to O2:
\begin{verbatim}
  #-
  S   a,b,c,d,e,f,g,h,i,j,k,l,m,n;
  L   G = (4*a^4+b+c+d + i^4 + g*n^3)^10 + 
          (a*h + e + f*i*j + g + h)^8 + (i + j + k + l + m + n)^12;
  L   H = G;
  Format O2;
  .sort
  #optimize G
  #write "Optimized with O2:"
  #write "Optimized with Horner scheme: `optimscheme_'"
  #write "Number of operations in output: `optimvalue_'"
  #clearoptimize
  .sort
  Format O4,saIter=1000; * use 1000 iterations for optimization
  #optimize H
  #write "Optimized with O4:"
  #write "Optimized with Horner scheme: `optimscheme_'"
  #write "Number of operations in output: `optimvalue_'"
  .end
\end{verbatim}
which gives the output:
\begin{verbatim}
Optimized with O2:
Optimized with Horner scheme: i,n,j,m,l,k,g,a,d,c,b,h,f,e
Number of operations in output: 2578

Optimized with O4:
Optimized with Horner scheme: m,h,k,a,l,e,n,g,j,c,f,b,i,d
Number of operations in output: 1937
\end{verbatim}

The preprocessor variable optimscheme\_ \index{optimscheme\_} gives the best Horner scheme that the
program found and the preprocessor optimvalue\_ \index{optimvalue\_} gives the number of
arithmetic operations in the resulting expression.

These levels of optimization refer to some default settings of all
controlling parameters. These default values are in
Tab.~\ref{tbl:defaults}. It is also possible to set each parameter
individually to fine-tune the optimization process. The parameters
that can be set are divided in several categories. First, it is
possible to set which Horner schemes\index{Horner scheme} are tried:

\begin{description}
\item[Horner=(Occurrence $|$ MCTS $|$ SA)] Determines whether an 
  occurrence order\index{occurrence order} Horner scheme is used, or
  whether MCTS\index{MCTS}\index{Monte Carlo tree search}, or Stochastic Local Search is employed to
  find Horner schemes.

\item[HornerDirection=(Forward $|$ Backward $|$ ForwardOrBackward $|$] \hfill
  {\bf ForwardAndBackward)}
  Forward makes that the MCTS search in the O3 option will 
  determine the outermost variables in the multivariate Horner scheme first 
  and then work its way inward.
  In the case of backward, the tree search determines the innermost variable 
  first. In some cases this can give much better results when there are 
  many common subexpressions involving a limited number of variables.
  ForwardOrBackward tries both of these
  schemes. ForwardAndBackward fills the order from both sides
  simultaneously, resulting in more options, but also a much larger
  search tree. If there are many variables, it could make the search tree 
  too large to obtain good results. \hfill \\
  When the option Horner=Occurrence is used the option backward will switch 
  to something called `anti-occurrence' which means that the most frequent 
  variable corresponds to the innermost brackets.
\end{description}

In the case of MCTS\index{MCTS}\index{Monte Carlo tree search} there are 
various parameters that can control the search process:

\begin{description}
\item[MCTSConstant=$<$\emph{value}$>$] 
  This sets the constant $C_P$ in the UCT formula that governs 
  the Monte Carlo tree search. It is supposed to be given as a real number 
  with a decimal point (no floating point notation that includes powers).
\item[MCTSNumExpand=$<$\emph{value}$>$] The number of times the tree
  is traversed and hence the number of times that a Horner scheme is
  constructed.
\item[MCTSNumKeep=$<$\emph{value}$>$] 
  During the MCTS procedure \FORM\ only tries to construct 
  a proper ordering for the Horner scheme, followed by a common subexpression 
  elimination in the style of the O1 option. The best `value' schemes are 
  remembered and for those a common subexpression elimination in the style of 
  the O2 option is done afterward. This second style elimination is far more 
  costly. In nearly all cases the best O2-style scheme is in the very few top 
  O1-style schemes.
\item[MCTSNumRepeat=$<$\emph{value}$>$] 
  Sometimes it is more advantageous to run
  a new tree search several times, each with a smaller number of
  expansions. This parameter tells how many times we will run with a
  new tree. The total number of tree traversals is the product of 
  MCTSNumRepeat and MCTSNumExpand.
\item[MCTSNumExpand=$<$\emph{value1*value2}$>$] 
  Makes \FORM\ to run `value1' trees, each with `value2' Horner scheme
  constructions. Hence this option is equivalent to the combination \hfill \\
  MCTSNumRepeat=$<$\emph{value1}$>$, MCTSNumExpand=$<$\emph{value2}$>$.
\item[MCTSTimeLimit=$<$\emph{value}$>$] The maximum time in seconds
  that is used when searching through the tree.
\item[MCTSDecayMode=$<$\emph{value}$>$] Determines how the $C_P$ parameter
in the UCT formula decreases:

\begin{center}
\begin{tabular}{|c|l|}
\hline
value & effect\\
\hline
0 & no decay\\
1 & linear decay with iteration number\\
2 & faster decay for the final iterations\\
3 & decrease with iteration number and with node depth\\
\hline
\end{tabular}
\end{center}
% 0 means there
%is no decay, 1 means a linear decay with iteration number, 2 a more
%agressive decay for the final iterations, and MCTSDecayMode=3 .
%The default value is 1.
\end{description}

For Stochastic Local Search the following parameters can be set:
\begin{description}
\item[saIter=$<$\emph{value}$>$] Number of optimization steps that will be performed. This has the most influence on the quality of the simplification. The default value is 1000.
\item[saMaxT=$<$\emph{value}$>$] Maximum temperature used in Simulated Annealing. The higher the temperature,
the more exploration occurs. The default value is 2000.
\item[saMinT=$<$\emph{value}$>$] Minimum temperature used in Simulated Annealing. The lower the temperature,
the more exploitation occurs. The default value is 1.
\end{description}
The cooling rate from saMaxT to saMinT is exponential in saIter. More information can be
found in the research papers.

The Horner methods generate a number of Horner schemes: one or two in
the case of occurrence order schemes, depending of the direction
parameter, and a number equal to MCTSNumKeep in the case of
MCTS. Next, for each stored Horner scheme other optimizations are
performed as determined by the following parameter:

\begin{description}
\item[Method=(None $|$ CSE $|$ Greedy $|$ CSEGreedy)] Determines what
method is used for optimizing the generated Horner schemes. 
CSE\index{CSE}\index{Common subexpression elimination} performs a simple 
common subexpression elimination and Greedy performs greedy 
optimizations\index{greedy optimizations} (see the paper for more 
explanations) which are more sophisticated versions of CSE's. CSEGreedy 
performs CSE followed by greedy optimizations; usually this is somewhat 
faster than just greedy optimizations, but it gives slightly worse results. 
The option None does nothing after applying the Horner scheme and is only 
useful for debugging purposes.
\end{description}

When the method of greedy optimizations is used, repeatedly all
potential optimizations are determined and a few of them are performed. The 
following parameters are used to tune the greedy method:
\begin{description}
\item[GreedyMaxPerc] The percentage of the possible optimizations that is
  performed.
\item[GreedyMinNum] The minimum number of possible optimizations that
  is performed.
\item[GreedyTimeLimit] The maximum time in seconds that is spent in
  the process of greedy optimization.
\end{description}

There are also two more general settings:
\begin{description}
\item[Stats=(On $|$ Off)] This parameter determines whether statistics
  of the optimization are shown. Statistics are printed in the format

{\tt *** STATS: original  1P 16M 5A : 23}

{\tt *** STATS: optimized 0P 10M 5A : 15}

in which P indicates power operations (at least a third power), M the 
number of multiplications and A the number of additions/subtractions. The 
last number is the total number of operations in which an $n$-th power counts 
as $n-1$ operations.
\item[TimeLimit=$<$\emph{value}$>$] This set both the MCTSTimeLimit
  and the GreedyTimeLimit to half of the given value.
\end{description}

Finally there are some parameters that are of a rather specialized nature. 
They can be used for debugging\index{debugging} purposes or in the case 
that one knows already what is the best Horner scheme. Their default values 
are Off.

\begin{description}
\item[DebugFlag=(On $|$ Off)] \label{optimdebugflag}
In the case that the value is On, the list of temporary variables is 
printed in reverse order with the string "id " in front. This makes 
them into a set of \FORM\ substitutions that undo the optimizations. One 
can use this for instance to make sure that the optimized code is identical 
to the original.
\item[PrintScheme=(On $|$ Off)]
This option (when On) will print the Horner scheme. That is the order in 
which the variables were taken outside parentheses.
\item[Scheme=(list of symbols)] The list should be enclosed by parentheses 
and the symbols should be separated by either blanks or comma's. This 
option will fix the Horner scheme\index{Horner scheme} to be used. One 
could for instance use the output of the PrintScheme option for this to 
avoid a lengthy search when a good order of the variables is already known. 
Things become a bit tricky when extra symbols are involved. One should make 
sure that their labelling is identical to when the scheme was created! When 
extra symbols are used in their array/vector notation, one needs to 
separate them by comma's, because blank spaces next to parentheses are 
eliminated by the preprocessor. If one specifies the wrong number of 
variables, the results can be quite unpredictable. At the moment of 
compilation \FORM\ does not know the variables that are actually used. The 
safe thing is to verify the actual variables with a testrun using the 
PrintScheme option in the O1 mode.
\end{description}

{ \small
\begin{table}[!ht]
\centering
\begin{tabular}{|l|c|c|c|c|}
\hline
                &     O1      &    O2       & O3 (default) & O4 (default) \\
 \hline
Horner          &  occurrence &  occurrence &   MCTS  &      SA     \\
HornerDirection &     OR      &      OR     &    OR   &      OR     \\
MCTSConstant    &    ---      &     ---     &   1.0   &     ---     \\
MCTSNumExpand   &    ---      &     ---     &  1000   &     ---     \\
MCTSNumKeep     &    ---      &     ---     &   10    &     ---     \\
MCTSNumRepeat   &    ---      &     ---     &    1    &     ---     \\
MCTSTimeLimit   &    ---      &     ---     &    0    &     ---     \\
MCTSDecayMode   &    ---      &     ---     &    1    &     ---     \\
saIter          &    ---      &     ---     &   ---   &     1000    \\
saMinT          &    ---      &     ---     &   ---   &      1      \\
saMaxT          &    ---      &     ---     &   ---   &     2000    \\
Method          &    cse      &    greedy   &  greedy &    greedy   \\
GreedyMinNum    &    ---      &     10      &   10    &     10      \\
GreedyMaxPerc   &    ---      &      5      &    5    &      5      \\
GreedyTimeLimit &    ---      &      0      &    0    &      0      \\
Stats           &    off      &     off     &   off   &     off     \\ 
TimeLimit       &     0       &      0      &    0    &      0      \\
\hline
\end{tabular}
\caption{Values for the various parameters in the predefined
  optimization levels. OR stands for ForwardOrBackward.}
\label{tbl:defaults}
\end{table}
} 

All options should be specified in a single format statement and be
separated either by commas or blank spaces. When
\verb|Format Optimize| is used, first the default settings are taken
and then the options that are specified overwrite them. It is allowed
to have the O1, O2, O3, O4 optimization specifications followed by
options. In that case the program first sets the values of those
specifications and then modifies according to what it encounters in
the rest of the statement.

